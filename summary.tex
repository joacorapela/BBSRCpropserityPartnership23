\subsection{NeuroGEARS}

NeuroGEARS is a UK technology company that helps research institutions adopt
state-of-the-art technologies to develop innovative biomedical research.  The
director of NeuroGEARS, Dr.~Goncalo Lopes, is the inventor, and his company is
the main supporter, of Bonsai, a reactive visual programming language that
powers thousands of experiments around the world.

\subsection{NeuroGEARS-SWC-GCNU collaboration}

NeuroGEARS has been working with the Sainsbury Wellcome Centre and the Gatsby
Computational Neuroscience Unit, both at University College London, for more
than two years developing hardware and software technology for unrestrained,
long-duration, naturalistic, closed-loop and intelligent experimentation in
neuroscience by combining its engineering expertise with the machine learning
and experimental neuroscience research experience of its academic partners.
Together they are opening a new chapter on neuroscience experimentation.

\subsection{Technology dissemination}

Open Ephys is a unique business model for disruptive technology dissemination
\ldots

We want to allow laboratories around the world to use the technology we are
creating with the SWC and Gatsby Unit following a similar business model as
that of Open Ephys with one important difference. While the technology of Open
Ephys is focused on hardware, our technology is focused on software \ldots

The technology that NeuroGEARS is developing in collaboration with the SWC and
the Gatsby Unit is specialised to support neuroscience experiments in rodents.
We offer to generalise this technology to support a wider range of experimental
setup and to distribute it to other research centres around
the world.

The potential of disseminating this technology are enormous. Academically,
this technology is allowing to probe systems in natural regimes that cannot
be studied with traditional simpler experiments, and to study these systems in
unprecedented detail. For example in the 24/7 experiments that we
are developing at the SWC we can study foraging in naturalistic environments
where we can control environmental experimental variables (e.g., food delivery)
with great precision, while we record and manipulate neural activity.

The business applications of these technologies are also very large. For
instance in the pharmaceutical industry, these technologies could allow
unprecedented efficiency for automatic drug testing \ldots Another application
appears in the domain of personalised healthcare \ldots

In addition, long-duration, naturalistic and close-loop experiments will
incentivize several adjacent industries, like those of computer storage (to
save the very large amounts of data produced by these experiments), or the
industry of physiological measurement devices (key components in our
experiments).

Furthermore, the novel datasets that our experiments are producing are also
incentivizing academic research in, for example, machine learning algorithms to
process real time time series.

\subsection{Proposal aim}

The current joint work between NeuroGEARS, the SWC and the Gatsby Unit is to
build infrastructure to perform long-duration, naturalistic, close-loop and
intelligent experiments in order to investigate a specifics problems in the
real systems neuroscience. We aim at distributing hardware, software and
machine learning technology to allow a wiser range of long-duration,
naturalistic and close-loop experiments.
