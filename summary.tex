\subsection{NeuroGEARS}

NeuroGEARS is an UK technology company that helps research institutions around the
world use state-of-the-art technologies to develop real-time and close-loop
innovative experiments. Goncalo Lopes, founder of NeuroGEARS is the creator of Bonsai, a
reactive visual programming language that powers thousands of experiments
around the world.

\subsection{NeuroGEARS-SWC-GCNU collaboration}

NeuoGEARS is currently partnering with the Sainsbury Wellcome Centre (SWC) for
Neural Circuits and Behaviour and with the Gatsby Computational Neuroscience
Unit (GCNU) to create hardware and software infrastructure to perform
long-duration (24/7), naturalistic, close-loop and intelligent mice foraging
experiments. Together they are opening a new chapter on experimentation. They
are solving key issues related to the storage and retrieval of the large amount
of data produced by these experiments, as well as creating new machine learning
algorithms to learn from this data.

\subsection{Technology dissemination}

NeuroGEARS offers to distribute the technology and expertise gained in this
joint work with the SWC and the Gatsby Unit to other research centres around
the world, and to create a spinoff company to use this technology in
personalised healthcare applications.

The potential of disseminating this technology are enormous. Academically,
this technology is allowing to probe systems in natural regimes that cannot
be studied with traditional simpler experiments, and to study these systems in
unprecedented detail. For example in the 24/7 experiments that we
are developing at the SWC we can study foraging in naturalistic environments
where we can control environmental experimental variables (e.g., food delivery)
with great precision, while we record and manipulate neural activity.

The business applications of these technologies are also very large. For
instance in the pharmaceutical industry, these technologies could allow
unprecedented efficiency for automatic drug testing \ldots Another application
appears in the domain of personalised healthcare \ldots

In addition, long-duration, naturalistic and close-loop experiments will
incentivize several adjacent industries, like those of computer storage (to
save the very large amounts of data produced by these experiments), or the
industry of physiological measurement devices (key components in our
experiments).

Furthermore, the novel datasets that our experiments are producing are also
incentivizing academic research in, for example, machine learning algorithms to
process real time time series.

\subsection{Proposal aims}

This proposal has two main aims. The current joint work between NeuroGEARS, the
SWC and the Gatsby Unit is to build infrastructure to perform long-duration,
naturalistic, close-loop and intelligent experiments in order to investigate a
specifics problems in the real systems neuroscience. The first aim of the
proposed work is to distribute hardware, software and machine learning
technology that could be useful to a wide range of long-duration, naturalistic
and close-loop experiments.

The second main goal of this proposal is to create a spinoff company to use the
disseminated technologies in the realm of personalised medicine.

